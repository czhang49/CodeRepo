\input{cs446.tex}
\usepackage{graphicx,amssymb,amsmath,url}
\usepackage{xcolor}
\sloppy
\newcommand{\ignore}[1]{}
\newcommand{\bx}{{\bf x}}
\newcommand{\bw}{{\bf w}}

\oddsidemargin 0in
\evensidemargin 0in
\textwidth 6.5in
\topmargin -0.5in
\textheight 9.0in

\newcommand{\bb}[1]{{\bf #1}}

\begin{document}

\assignment{Fall 2014}{4}{October $8^{th}$, $2014$}{October $17^{th}$, $2014$}


\begin{footnotesize}
\begin{itemize}
\item Feel free to talk to other members of the class in doing the homework.  I am
more concerned that you learn how to solve the problem than that you
demonstrate that you solved it entirely on your own.  You should, however,
write down your solution yourself.  Please try to keep the solution brief and
clear.

\item Please use Piazza first if you have questions about the homework.
  Also feel free to send us e-mails and come to office hours.

\item Please, no handwritten solutions. You will submit your solution manuscript as a single pdf file.

\item Please present your algorithms in both pseudocode and English.  That is, give
a precise formulation of your algorithm as pseudocode and {\em also} explain
in one or two concise paragraphs what your algorithm does.  Be aware that
pseudocode is much simpler and more abstract than real code. 

\item The homework is due at 11:59 PM on the due date. We will be using
Compass for collecting the homework assignments. Please submit your solution manuscript as a pdf file via Compass
(\texttt{http://compass2g.illinois.edu}). Please do NOT hand in a hard copy of your write-up.
Contact the TAs if you are having technical difficulties in
submitting the assignment.

\item \textcolor{red}{You cannot utilize the late submission credit hours for this problem set.}

\item No code is needed for any of these problems. You can do the
calculations however you please. You need to turn in only the report.

\end{itemize}
\end{footnotesize}
\begin{enumerate}
\item \textbf{[VC Dimension - 30 points]}
\begin{enumerate}
  \item \textbf{[15 points]}  Assume that all examples are points in two-dimensional space,
  i.e.~$\bb{x} = \langle x_1, x_2\rangle \in \mathbb{R}^2$.
Consider the concept space of circles with
  arbitrary origin and radius. Hence, a concept $h \in \bb{H}$ has three
  parameters, $r \in \mathbb{R}^+$ and the two coordinates of the center $\bb{x}_0 \in \mathbb{R}^2$.
  An example $\bb{x} \in \mathbb{R}^2$ is labeled as positive by $h$
  if and only if $\bb{x}$ lies within a circle of radius $r$ centered at $\bb{x}_0$ and of
  radius $r$, i.e.~$\|\bb{x} - \bb{x}_0\| < r$. Give the VC dimension of $\bb{H}$ and prove that your answer is correct.

  \item \textbf{[15 points]} Consider the concept space of the union of $k$ disjoint intervals in a real line.
  Hence, a concept $h \in \bb{H}$ is represented by $2k$ parameters $a_1 < b_1 < a_2 < b_2 < \ldots < a_k < b_k$.
  An example $x \in \mathbb{R}$ is labeled as positive by $h$
  if and only if $x$ lies in one of the intervals $[a_p,b_p], p\in \{1,2,\ldots,k\}$. Give the VC dimension of $\bb{H}$ and  prove that your answer is correct.

  \end{enumerate}

{\bf Grading note:} You will not get any points without proper justification of your answer.

\item \textbf{[Decision Lists - 40 points]}

  In this problem, we are going to learn the class of $k$-decision
  lists.  A decision list is an ordered sequence of if-then-else
  statements. The sequence of if-then-else conditions are tested in
  order, and the answer associated with the first satisfied condition is
  returned. The class of $k$-decision lists is a subset of the class of all 
  decision lists, where the statements in each rule are of a bounded size.
  Formally, for a fixed $k$, we define:

\begin{definition}
  A \textbf{\underline{\em $k$-decision list}} over the variables $x_1, \ldots, x_n$
  is an ordered sequence $L=(c_1, b_1), \ldots, (c_\ell, b_\ell)$ and
  a bit $b$, in which each $c_i$ is a conjunction of at most $k$
  literals over $x_1,\ldots, x_n$.  The bit $b_i$ is referred to as the bit
  \underline{\em associated} with condition $c_{i}$, and $b$ is called the \underline{\em default} value.  For any input $\bb{x} \in \{0,
  1\}^{n}$, $L(\bb{x})$ is defined to take the value $b_{j}$, where $j \in \{1, \ell\}$ is the
  smallest index satisfying $c_j(\bb{x})=1$; if no such index exists,
  then $L(\bb{x})=b$.

  We denote by \textbf{\underline{\em $k$-DL}} the class of concepts that can be
  represented by a $k$-decision list.
\end{definition}

Figure~\ref{fig:decision_list} shows an example of a $2$-decision list over six variables, $x_1, \ldots, x_6$. For this decision list, $L(\bb{x}_1 = \langle 0,1,1,0,0,1\rangle) = 1$ and $L(\bb{x}_2 = \langle 1,0,0,1,0,0\rangle) = 0$,
where in the case of $x_1$ no condition is satisfied, so the labels is determined by the default bit, and for $x_2$, the second condition is satisfied. 

\begin{figure}[h]
\begin{center}
\includegraphics[width=1.35in]{dec-list.jpg}
\caption{A $2$-decision list.}
\label{fig:decision_list}
\end{center}
\end{figure}



\begin{enumerate}
\item \textbf{[5 points]} Show that if a concept $c$ can be represented as a $k$-decision
  list so can its complement, $\neg c$. You can show this by providing
  a $k$-decision list that represents $\neg c$, given $c =
  \langle (c_1,b_1), \ldots, (c_\ell,b_\ell), b\rangle$.

\item \textbf{[10 points]} Show that
$$k\textrm{-DNF} \cup k\textrm{-CNF} \subseteq k\textrm{-DL}$$

\item \textbf{[15 points]} Let $S$ be a sample data set that is \emph{consistent} with some
  $k$-decision list. Provide an algorithm to construct a $k$-decision
  list that is consistent with $S$. Prove the correctness of your
  algorithm.

\item \textbf{[10 points]} Use  Occam's Razor to show: \\
  For any constant $k \geq 1$, the class of $k$-decision lists is
  efficiently PAC-learnable.


\end{enumerate}


\item \textbf{[Constructing Kernels - 30 points]}

  For this problem, we wish to learn a Boolean function represented as a DNF (not necessarily
  monotone) using kernel Perceptron. For this problem, assume that the size of each term in the DNF
  is bounded by $k$, i.e., the number of literals in each term of the DNF is 
  between $1$ to $k$. In order to complete this task,
  we will first define a kernel that maps an example $\bb{x} \in \{0, 1\}^n$ into a
  new space of conjunctions of \textbf{upto} $k$ different literals from the
  $n$-dimensional space. Then, we will use the kernel Perceptron to perform our
  learning task. 
 
  \begin{enumerate}
  \item \textbf{[10 points]} Define a kernel $K(\bb{x}_1, \bb{x}_2) = \sum_{c \in C} c(\bb{x}_1) c(\bb{x}_2)$,
    where $C$ is a family of conjunctions containing upto $k$ different
    literals, and $c(\bb{x}) \in \{0, 1\}$ is the value of $c$ when evaluated
    on example $\bb{x}$. Show that $K(x_1,x_2)$ can be computed in time that is linear in $n$. [Hint: It would be useful to think about the case where each term of the 
  DNF is \emph{exactly} of $k$ literals and then generalize to the case where each term of the DNF 
  is from $1$ to $k$ literals.]


  \item \textbf{[10 points]} Write explicitly the kernel Perceptron algorithm that uses
    your kernel.

  \item \textbf{[10 points]} State a bound on the number of mistakes the kernel Perceptron algorithm will
    make on a sample of examples consistent with a function in this class.
[Hint: In class, we proved a theorem by Novikoff; use this bound and estimate the two constants in it for this specific case. 
When doing it observe that you are no longer learning in the original $n$ dimensional space, but rather in a blown-up space 
that has a different dimensionality.]
\end{enumerate}


\end{enumerate}

\end{document}


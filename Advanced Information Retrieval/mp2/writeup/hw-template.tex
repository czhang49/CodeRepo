\documentclass[12pt,letterpaper]{article}

\newenvironment{proof}{\noindent{\bf Proof:}}{\qed\bigskip}

\newtheorem{theorem}{Theorem}
\newtheorem{corollary}{Corollary}
\newtheorem{lemma}{Lemma} 
\newtheorem{claim}{Claim}
\newtheorem{fact}{Fact}
\newtheorem{definition}{Definition}
\newtheorem{assumption}{Assumption}
\newtheorem{observation}{Observation}
\newtheorem{example}{Example}
\newcommand{\qed}{\rule{7pt}{7pt}}

\newcommand{\assignment}[4]{
\thispagestyle{plain} 
\newpage
\setcounter{page}{1}
\noindent
\begin{center}
\framebox{ \vbox{ \hbox to 6.28in
{\bf CS446: Machine Learning \hfill #1}
\vspace{4mm}
\hbox to 6.28in
{\hspace{2.5in}\large\mbox{Problem Set #2}}
\vspace{4mm}
\hbox to 6.28in
{{\it Handed Out: #3 \hfill Due: #4}}
}}
\end{center}
}

\newcommand{\solution}[4]{
\thispagestyle{plain} 
\newpage
\setcounter{page}{1}
\noindent
\begin{center}
\framebox{ \vbox{ \hbox to 6.28in
{\bf CS446: Machine Learning \hfill #4}
\vspace{4mm}
\hbox to 6.28in
{\hspace{2.5in}\large\mbox{Problem Set #3}}
\vspace{4mm}
\hbox to 6.28in
{#1 \hfill {\it Handed In: #2}}
}}
\end{center}
\markright{#1}
}

\newenvironment{algorithm}
{\begin{center}
\begin{tabular}{|l|}
\hline
\begin{minipage}{1in}
\begin{tabbing}
\quad\=\qquad\=\qquad\=\qquad\=\qquad\=\qquad\=\qquad\=\kill}
{\end{tabbing}
\end{minipage} \\
\hline
\end{tabular}
\end{center}}

\def\Comment#1{\textsf{\textsl{$\langle\!\langle$#1\/$\rangle\!\rangle$}}}



\oddsidemargin 0in
\evensidemargin 0in
\textwidth 6.5in
\topmargin -0.5in
\textheight 9.0in

\begin{document}

\solution{chen zhang, NetId: czhang49}{\today}{2}{Fall 2016}
% Fill in the above, for example, as follows:
% \solution{Joe Smith}{\today}{1}{Fall 2012}

\pagestyle{myheadings}  % Leave this command alone

\section{Entropy}
\subsection{a}
\begin{equation}
H(W) = -\sum_{x\in\Omega}P(x)\log_2 P(x)
\end{equation}
The minimum happens when $P(W=w_k)=1$, and $P(W=w_i)=0$ for all $i\neq k$. Then equation (1) becomes
\begin{equation}
H(W) = -P(W=w_k)\log_2 P(W=w_k)=0
\end{equation}
Theoretical maximum happens when W has a even distribution among all possible N words. Then we have
\begin{equation}
H(W) = -\sum_{x\in\Omega}P(x)\log_2 P(x)=-N\frac{1}{N}\log_2 \frac{1}{N}=\log_2 N
\end{equation}

\subsection{b}
Article1 -- minimum : $w_1$, $w_1$, $w_1$, $w_1$.
\\
Article2 -- maximum : $w_1$, $w_2$, $w_3$, $w_4$, $w_5$, $w_6$.

\subsection{c}
Since for $A_1$ and $A_2$, we all have $H(W)=0$, then $A_1$ and $A_2$ are constructed with only one word, respectively. So when concatenating  $A_1$ and $A_2$, maximum entropy happens if $A_1$ and $A_2$ are constructed with different words, and with equal length(minimum entropy happens when they are constructed with the same word). \\
Article $A_1$: $w_1$, $w_1$, $w_1$, $w_1$.\\
Article $A_2$: $w_2$, $w_2$, $w_2$, $w_2$.

\section{Conditional Entropy and Mutual Information}
\subsection{a}
\begin{equation}
H(X|X) = -\sum_{X\in\Omega}P(X|X)\log_2 P(X|X),
\end{equation}
since $P(X|X)=1$, then above equation becomes 
\begin{equation}
H(X|X) = -N\times1\times\log_2 1=0
\end{equation}

\subsection{b}
We have the following equation:
\begin{equation}
I(X;Y) = -\sum_{x,y}P(x,y)\log_2 \frac{P(x,y)}{p(x)p(y)},
\end{equation}
if $x$ and $y$ are mutually independent, then we have $p(x,y)=p(x)p(y)$, plug in the above equation, we have
\begin{equation}
I(X;Y) = -\sum_{x,y}P(x)P(y)\log_2 \frac{P(x)P(y)}{p(x)p(y)}= -\sum_{x,y}P(x)P(y)\log_2 1=0.
\end{equation}
We can also understand it in this way, since $I(X;Y) = H(X) - H(X|Y)$, if $x$ and $y$ are mutually independent, then $H(X|Y)=H(X)$, and then $I(X;Y) =0$.


\section{Mutual Information of Words (Programming Exercise)}
\subsection{a}
\begin{equation}
P(X_A=0,X_B=1)=\frac{N_B-N_{AB}}{N}
\end{equation}
\begin{equation}
P(X_A=0,X_B=0)=\frac{N-N_A-N_B+N_{AB}}{N}
\end{equation}

\subsection{b}
paper january 181 \\
language programming 153 \\
january time 150 \\
systems january 149 \\
program january 149 \\
data january 142 \\
presented january 141 \\
programming january 139 \\
program programs 133 \\
method january 125 \\

\subsection{c}
\subsubsection{i}
program storage \\
programming method \\
programming time\\
data january \\
paper presents \\
language time \\
paper describes \\
intelligence artificial \\
output input \\
jr thacher \\
\\
Many of the words are different, only ( data january ) is also in part (b).

\subsubsection{ii}
method\\
time \\
paging \\
polynomial \\
distribution \\
\\
I think it makes sense, because programming is associated with these terms . (e.g. method in programming).
\end{document}


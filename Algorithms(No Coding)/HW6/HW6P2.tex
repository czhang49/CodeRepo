% ---------
%  Compile with "pdflatex hw0".
% --------
%!TEX TS-program = pdflatex
%!TEX encoding = UTF-8 Unicode

\documentclass[11pt]{article}
\usepackage{jeffe,handout,graphicx}
\usepackage[utf8]{inputenc}		% Allow some non-ASCII Unicode in source
\usepackage{amsmath}
%  Redefine suits
\usepackage{pifont}
\def\Spade{\text{\ding{171}}}
\def\Heart{\text{\textcolor{Red}{\ding{170}}}}
\def\Diamond{\text{\textcolor{Red}{\ding{169}}}}
\def\Club{\text{\ding{168}}}

% =========================================================
%   Define common stuff for solution headers
% =========================================================
\Class{New CS 473}
\Semester{Spring 2015}
%\Section{}

% =========================================================
\begin{document}

% ---------------------------------------------------------
% Change authors for all future solutions
\AuthorOne{Arash Khatibi}{khatibi2}
\AuthorTwo{Liang Tao}{ltao3}
\AuthorThree{Chen Zhang}{czhang49}
\HomeworkHeader{6}{2}


\section{Part A}
Define the following variables:

\noindent T: A binary tree T.

\noindent P(Q): The length of a random root-to-leaf path in the binary tree T.

\noindent  When we meld two arbitrary heap-ordered binary tree $Q_1$ and $Q_2$. The worst case is that the algorithm will trace down both $Q_1$ and $Q_2$. The best case is that the algorithm will only trace down one tree (when all the elements in $Q_1$ has smaller or larger priority than $Q_2$). Each time an operation is carried out, we will trace down one step on $Q_1$ or $Q_2$, either to the left or to the right. Thus we know that the maximum step we trace down is the summation of the length of the two trees, which is $P(Q_1)+P(Q_2)$. Now we will prove that 
\[
E[P(Q_1)+P(Q_2)]=O(\log{n}).
\] 
It suffice to show that for a tree T with n nodes, $E[P(T)]=O(\log{n})$, since if we remove the root of T, it will readily become two trees with a total number of nodes equal to n-1.

\noindent Now we claim that $E[P(n)]\leq \log{(n+1)}$. If $n=0$, this claim is readily satisfied. Otherwise, suppose that the left and right subtrees of T contains $l$ and $r$ nodes, respectively. Notice that $l < n$ and $r < n$. With strong induction hypothesis, we have $E[P(l)]\leq \log{(l+1)}$ and $E[P(r)]\leq \log{(r+1)}$. Thus we can have the following equations:
\begin{align*}
\dfrac{1}{2}(E[P(l)]+E[P(r)])
& \leq \dfrac{1}{2}(\log{(l+1)}+\log{(r+1)})\\
& =\log{(2\sqrt{(l+1)(r+1)})}\\
& \leq \log{(l+1)+(r+1)} \\
& \leq \log{(n+1)}
\end{align*}

Thus the strong induction proof holds true and we have that $E[P(Q_1)+P(Q_2)]=O(\log{n})$. 

\section{Part B}
Define the following variables:

\noindent $|\gamma|$ : The length of path $\gamma$.

\noindent $h_Q$: The length of a path from root to leaf. 

\noindent $\Gamma$: The set of all paths from the root to leaf with length exceeding $(c+1)\log{n}$.

Note that to reach for any leaf from the root via any fixed path $\gamma$, the probability is $2^{-|\gamma|}$. Suppose that $\Gamma$ is the set of all paths from the root to leaf with length exceeding $(c+1)\log{n}$, we have the following relation:
\[
Pr[h_Q>(c+1)\log{n}]=\sum\limits_{\gamma\in\Gamma}2^{-|\gamma|}<\sum\limits_{\gamma\in\Gamma}2^{-(c+1)\log{n}}=|\Gamma|n^{-(c+1)}\leq n^{-c}
\]

Thus we see that we have a exponentially decaying probability of length of path exceeding $(c+1)\log{n}$ with the number of node $n$. Note the running time of Meld($Q_1$,$Q_2$) is $P(Q_1)+P(Q_2)$. This proves that the probability that we have running time of Meld($Q_1$,$Q_2$) having $O(\log{n})$ is large.

\section{Part C}
\noindent MakeQueue: Create an empty set. This is O(1) time.\newline
\noindent FindMin(Q): Return the $Key(Q)$, which is the root of the tree, this takes O(1) time.\newline
\noindent DeleteMin(Q): Return Meld(left(Q),right(Q)).\newline
\noindent Insert(Q,x): Make the new element itself a new tree which takes O(1) time and return Meld(Q,x). \newline
\noindent Delete(Q,x): Let $Q_x$ be the tree with root at x, replace $Q_x$ with DeleteMin($Q_x$) which calls Meld() once and then return Q.\newline
\noindent DecreasePriority(Q,x,y): Let $Q_x$ be the tree with root at x, detach it from the parent, change the value of $Key(x)$ to y and return Meld(Q,$Q_x$) 



\end{document}
